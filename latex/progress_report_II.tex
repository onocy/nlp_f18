
\documentclass[11pt,a4paper]{article}
    \usepackage[hyperref]{naaclhlt2019}
    \usepackage{times}
    \usepackage{latexsym}
    \usepackage{graphicx}
    
    \usepackage{url}
    
    \aclfinalcopy 
    
    \title{Lyrical Distinction}

    \author{Jesse Bartola \\
      {\tt jbartola@} \\\And
      Phillip Michalowski \\
      {\tt pmichalowski@ } \\\And
      Nischal Tamang \\
      {\tt ntamang@} \\\And
      Max Berezin \\
      {\tt mberezin@} \\}
      
    % \setlength\textwidth{16.0cm}
    \date{}
    
    \begin{document}
    \maketitle
    
    \section{Problem statement}
    Now that you've had some time to work on the project, please clearly and concisely summarize the goal of your project and its motivations in 1-2 paragraphs. If your project has changed since you submitted your proposal, please explain what prompted the change.
    
    Some general guidelines for the progress report: 4-6 pages, due Nov. 16. The report must contain at least one table or graph that conveys numerical information – for example, statistics about your data or annotations, accuracy or other results of running an algorithm on the data, or something else. We highly recommend using this LaTeX template to write your report. Note that some sections may not be relevant for your project; feel free to delete them and add other sections if you wish! If you're unsure whether to include something or not, please see me in office hours / after class or post on Piazza / the anonymous form.
    
    \section{Your dataset}
    The most important rule of NLP: look at your data! Provide us with examples from your dataset. Explain what properties of the data make your task challenging. Report the source of the dataset, its basic statistics (e.g., size, number of words/sentences/documents) and some other statistics that are specifically relevant to your task. 
    
    \section{Annotation}
    If your project involves annotation, you may have started a pilot annotation experiment, annotating a few dozen or few hundred examples. What major issues have come up? Do you and your project partners agree or disagree on examples? (At this stage, qualitative findings about these questions are fine.)
    
    \section{Baselines}
    Run some sort of NLP algorithm — classifier, parser, etc. — on the data, and report its result. Additionally, explain how it works, and what hyperparameters you are using! If you are using a ready-made dataset, you should define a train/test split, and you should have at least one accuracy number to report at this point.
    
    \section{Error analysis}
    What kinds of inputs does your baseline fail at? Are there any semantic or syntactic commonalities between the examples it fails on? You may want to do a manual error analysis (e.g., annotate around 50 failed examples for various properties). How does (or will) your proposed approach fix these issues?
    
    \section{Your approach}
    What have you accomplished regarding your proposed approach? How does it work? Is your implementation already complete or in progress? What libraries are you currently using? What kind of computers are you running your experiments on? Are there any issues that you're stuck on? Be specific!!!
    
    \section{Timeline for the rest of the project}
    You now have a better idea of how much you can accomplish in the rest of the semester. Lay out the major items you want to accomplish. Provide a timeline to finish them by.
    
    
    \bibliographystyle{apalike}
    \footnotesize
    \bibliography{yourbib}
    
    
    \end{document}
    